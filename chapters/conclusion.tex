\chapter{Conclusiones}
\begin{flushright} 
I was sick - sick unto death with that long agony...\\
\emph{- The Pit and the Pendulum-}\\
\emph{Edgar Allan Poe}
\end{flushright}

Este trabajo consistió en la modelación matemática, así como la construcción de una bocina direccional. Partiendo desde las ecuaciones básicas de la acústica, y con base en un esquema de orden apropiado y sólido, se construyó la ecuación de onda que describe los fenómenos de difracción, absorción y no linealidad (la ecuación KZK). A partir de ahí, los supuestos apropiados fueron adoptados para terminar con un modelo que sirviera para simular el fenómeno y tener un marco de referencia; pues la siguiente parte del trabajo consistió en la experimentación sobre el fenómeno.\medskip\\
Como cierre para el trabajo, se exhiben los comentarios acerca de los experimentos realizados en el capítulo anterior. ...
\section{Comment on your Results.} 
El trabajo consistió en el desarrollo de un modelo que explicara la interacción no lineal en el campo cercano de las ondas ultrasónicas (de AM), hasta el modelo KZK que describe de forma precisa la propagación sonora que considera los efectos de difracción, absorción y no linealidad del medio.
\subsection{Experimentos sobre la funcionalidad común del PA}
Para los primeros experimentos, se pudo observar que, a pesar de que las mediciones no se hicieron en una cámara anecóica, éstas se muestran bastante cercanas a las predicciones hechas en las simulaciones. La figura ....
\subsection{Experimentos de cancelación sonora}
Estos experimentos, no pretendían establecer un sistema de control de ruido como el mostrado en \cite{feasibility}, sino buscar algún indicio de que existe algún tipo de interacción entre las ondas ultrasónicas y las audibles. Los experimentos realizados en esta etapa fueron mucho más sensibles al ruido externo, y las condiciones no ideales fueron determinantes en el resultado, por lo menos de los experimentos de interacción de las ondas con varias frecuencias....
\section{Aplications of your work}
Las aplicaciones de la acústica no lineal no se limitan al desarrollo de un arreglo paramétrico. Como se vio en el capítulo \ref{chapter.pa}, éste es únicamente una aplicación derivada de una condición muy específica del medio en el que se trabaja. El rango de aplicaciones derivadas de la acústica no lineal, y en particular los usos que tiene la ecuación KZK, es muy amplio y depende en gran parte de ...
\subsubsection{Ethics ?? }
Es claro que el progreso científico no debe estar exento de la responsabilidad de advertir sobre las consecuencias del mal manejo de las tecnologías desarrolladas. En el caso presentado en este trabajo, e....
\section{Future work}
El trabajo desarrollado en esta tesis tiene una trayectoria hacia el futuro muy bien definida y con un alto potencial de desarrollo. En términos del estudio del arreglo paramétrico en sus funcionalidades comunes,...